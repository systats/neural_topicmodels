
%<!-- Titleseite -->
\thispagestyle{empty}
\hspace*{.5\textwidth}
	 	\begin{flushright}
		      \vspace{.2cm}
         	{\includegraphics[width=6cm]{images/logo_stuttgart.jpg}\\
         	 Institute for Social Sciences\\
           Department of Political Theory and\\ 
	         Empirical Reasearch of Democracy\par}
   	\end{flushright}
	   
\vspace*{20mm}

Master's thesis submitted in \newline
partial fulfillment of the requirements \newline
for the degree of \textbf{Master of Arts (M. A.)}

\vspace*{10mm}




% An Investigation of Whether Parliaments Matter for the Legislative Process?

\begin{flushleft}
	{\LARGE \bfseries The Good, the Bad and the Ugly: the Strategic use of Sentiments in the German Parliament \par}
\end{flushleft}

\vspace*{10mm}
 
% \begin{minipage}[t]{0.5\textwidth}
%   \textbf{Rebecca Litauer}\\
%   Email: \href{mailto:rebecca.litauer@t-online.de}{rebecca.litauer@t-online.de}\\
%   Matr.: 2809173\\[0.5cm]
% \end{minipage}
% \begin{minipage}[t]{0.5\textwidth}
%   \textbf{Simon Roth}\\
%   Email: \href{mailto:nomis.roth@gmx.net}{nomis.roth@gmx.net}\\
%   Matr.: 2805614
% \end{minipage}

Author: Simon Roth \newline
Matr.: 2805614 \newline
Email: \href{mailto:nomis.roth@gmx.net}{nomis.roth@gmx.net}

\vspace*{10mm}

\textbf{Supervisors:} \newline 

Prof. André Bächtiger \newline
Prof. Patrick Bernhagen \newline

\vspace*{10mm}

Date of Submission: October 15, 2018


% \textbf{Abstract:} In this study we explain extreme right-wing voting behaviour in the countries of the European Union and Norway from a micro and macro perspective. Using a multidisci- plinary multilevel approach, we take into account individual-level social background char- acteristics and public opinion alongside country characteristics and characteristics of extreme right-wing parties themselves. By making use of large-scale survey data (N = 49,801) together with country-level statistics and expert survey data, we are able to explain extreme right-wing voting behaviour from this multilevel perspective. Our results show that cross- national differences in support of extreme right-wing parties are particularly due to dif- ferences in public opinion on immigration and democracy, the number of non-Western residents in a country and, above all, to party characteristics of the extreme right-wing parties themselves.


\clearpage
%\clearpage
%
%%<!-- Inhaltsverzeichnisse -->
%\thispagestyle{empty}
%\setstretch{1.15}
%\tableofcontents
%\listoffigures
%
%\clearpage
%\setstretch{1.44}
%<!-- \onehalfspacing -->