
%<!-- Titleseite -->
\thispagestyle{empty}

\vspace*{20mm}

\begin{center}
	{\LARGE \bfseries An Autoencoder Approach to Topicmodeling\par}
\end{center}

\vspace*{20mm}

\begin{minipage}[t]{0.5\textwidth}
  Simon Roth \\
  \href{mailto:s.roth@university-konstanz.de}{s.roth@university-konstanz.de}\\
  Student ID: 01950332 \\
\end{minipage}
\begin{minipage}[t]{0.5\textwidth}
	\begin{flushright}
    University of Konstanz\\
    Computational Social Science\\
    Dr. Karsten Donnay\\ 
	  Winter Term 2017/2018
  \end{flushright}
\end{minipage}

% \begin{minipage}[t]{0.5\textwidth}
%   \textbf{Rebecca Litauer}\\
%   Email: \href{mailto:rebecca.litauer@t-online.de}{rebecca.litauer@t-online.de}\\
%   Matr.: 2809173\\[0.5cm]
% \end{minipage}
% \begin{minipage}[t]{0.5\textwidth}
%   \textbf{Simon Roth}\\
%   Email: \href{mailto:nomis.roth@gmx.net}{nomis.roth@gmx.net}\\
%   Matr.: 2805614
% \end{minipage}

\vspace*{20mm}

%Date of Submission: October 15, 2018

\textbf{Abstract:} This paper aims to explore latent class models and their application to texts from social sciences. Latent Dirichlet Allocation (LDA) is a popular model to compress texts into topic vectors in an unsupervised fashion. More recently autoencoders (AE) have been proposed to tackle the problem of high dimensionality of text with the power of deep learning. Variational autoencoder (VAE) brings the world of Bayesian inference to high capacity neural networks. As the experimental part shows, VAE systematically outperforms LDA on several information retrieval tasks and establishes visually the most information-rich latent space of all benchmarked models. High quality document encodings are able to accuartly detect toxic language, fake news or subjects of scientific publications.